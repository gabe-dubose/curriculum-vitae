\documentclass{article}

\usepackage[letterpaper, portrait, margin=1in]{geometry}
\usepackage{parskip}


\newcommand{\qed}{\hfill$\Box$}
\newcommand{\closeproof}{\mbox{\hspace{1em}\rule{.45em}{.45em}}}
\begin{document}

\begin{flushleft}
{\Huge James G. DuBose} \rule{16.51cm}{0.4pt} \\
\end{flushleft}
\noindent
Research Specialist \hfill 1639 Briarcliff Road NE, Apt 8, Atlanta, GA 30306\\
Department of Biology \hfill gabe.dubose.sci@gmail.com\\
Emory University \hfill https://gabe-dubose.github.io/ \\
\end{flushleft}

\begin{flushleft}
{\Large Education} \rule{16.51cm}{0.4pt}\\
\end{flushleft}

\noindent Georgia Institute of Technology \hfill December 2022 \\
\noindent M.S. in Bioinformatics\\
\\
\noindent
\noindent University of Central Arkansas \hfill May 2021 \\
\noindent B.S. in Biology \\
\noindent Minors: Chemistry and Anthropology\\

\begin{flushleft}
{\Large Research} \rule{16.51cm}{0.4pt}\\
\end{flushleft}
\textbf{Georgia Institute of Technology} \hfill January 2022 – December 2022 \\
\textbf{Principal Investigator}: Dr. William Ratcliff \\
\textbf{Synopsis}: During my master's, I focused on developing new bioinformatic and 
statistical approaches to analyze variants in population genomic data. 
Recent work has shown that many types of variants that were presumed neutral can have large effects. My focus has been developing statistical 
models that can account for this variability of mutational effects and developing accompanying bioinformatic tools that allow others to implement 
these approaches as well. My work in the Ratcliff lab has primarily focused on using yeasts (\emph{Saccharomyces cerevisiae}) to develop these approaches, 
with the intent that they will be applicable to other organisms as well. \\
\\
\textbf{Georgia Institute of Technology} \hfill August 2021 – April 2022 \\
\textbf{Principal Investigator}: Dr. Kostas Konstantinidis \\
\textbf{Synopsis}: In the Environmental Microbial Genomic Laboratory I combined transcriptomic and proteomic approaches to study the physiology of 
obligate halogen-respiring bacteria. These soil-dwelling bacteria subsist on a rather rare soil compound and grow significantly slower than other bacteria. 
My objective was to understand what physiological adaptations allow for this lifestyle. Additionally, such bacteria have been implied in bio-remediation 
efforts, as halogen pollutants pose large risks for the environment.\\

\noindent \textbf{University of Central Arkansas} \hfill January 2019 – April 2021\\
\textbf{Principal Investigator}: Dr. Tammy Haselkorn\\
\textbf{Thesis}: \emph{The Ecological and Evolutionary Dynamics of Social Amoeba Microbiomes and Key Symbionts} \\
\textbf{Synopsis}: As an undergraduate student I studied the symbiotic interaction between soil dwelling amoebae, their persistent bacterial symbionts, 
and their transient bacterial microbiota. My thesis work focused on understanding the key ecological drivers of these symbiotic interactions, 
including how various symbionts are transmitted, their relevance in shaping their host microbiota, and how abiotic soil factors influence the stability of their interactions. 
I have continued working with Dr. Haselkorn to study how abiotic and biotic soil factors influence amoeba-symbiont interactions and amoeba ecology in general. 
Recently, we have also started investigating how host genetics and diet contribute to shaping the gut microbiota of flies (\emph{Drosophila sp.}) 
that feed exclusively on cactus, aiming to gain a broader understanding of how host diet, genetics, and population structure influence their gut microbiota.\\
\\ \\
\begin{flushleft}
{\Large Publications} \rule{16.51cm}{0.4pt}\\
\end{flushleft}
\hangindent=0.7cm \textbf{J.G. DuBose}, M. Robeson, M. Hoogshagen, H. Olsen, T.S. Haselkorn. 2022. The complexities of inferring symbiont function: Paraburkholderia symbiont dynamics in social amoeba populations and its impact on the amoeba microbiome. \emph{Applied and Environmental Microbiology}. doi: 10.1128/aem.01285-22 \\

\hangindent=0.7cm J. T. Pentz, K. MacGillivray, \textbf{J.G. DuBose}, P. L. Conlin, E. Reinhardt, E. Libby, W. C. Ratcliff. 2022. Evolutionary consequences of nascent multicellular life cycles. \emph{bioRxiv}. doi: 10.1101/2022.07.21.500805 - In review at \emph{eLife} \\

\hangindent=0.7cm \textbf{J.G. DuBose}., Y. Li, G.O. Bozdag, W.C. Ratcliff. 2023. Varanus: A Scalable Pythonic Variant Annotation Program. \emph{GitHub Repository}. - Manuscript in Preparation \\

\begin{flushleft}
{\Large Talks and Presentations} \rule{16.51cm}{0.4pt}\\
\end{flushleft}
\textbf{ASM South Central Branch 2022}, Poster \hfill October 27, 2022\\
Thomas B. Crook, \textbf{James G. DuBose}, Luciano Matzkin, Tamara S. Haselkorn. \emph{Comparative Microbiome Analysis of Cactophilic Drosophila Species}\\
\\
\textbf{Arkansas INBRE 2022}, Poster \hfill October 21, 2022\\
Thomas B. Crook, \textbf{James G. DuBose}, Luciano Matzkin, Tamara S. Haselkorn. \emph{The Microbiota of Naturally Acquired Cactophilic Drosophila Species}\\
\\
\textbf{Evolution 2021}, Talk \hfill June 23, 2021\\
\textbf{James G. DuBose}, Tamara S. Haselkorn. \emph{The transmission and diversity of Paraburkholderia in natural D. discoideum populations and its impact on the D. discoideum microbiome}\\
\\
\textbf{Asilomar 2021}, Talk \hfill January 08, 2021\\
\textbf{James G. DuBose}, Tamara S. Haselkorn. \emph{The Domination of Paraburkholderia in the Social Amoeba D. discoideum microbiome and its Impact on the Ecological Relevance of the Farming Symbiosis}\\
\\
\textbf{Arkansas INBRE 2020}, Talk \hfill November 06, 2020\\
\textbf{James G. DuBose}, Tamara S. Haselkorn. \emph{The Genetic Diversity of Bacterial Symbionts in Dictyostelium discoideum Social Amoeba and Their Effect on the Amoeba Microbiome}\\
\\
\textbf{ASM Microbe}, Poster \hfill July 2020\\
\textbf{James G. DuBose}, Hunter Olsen, Tamara S. Haselkorn. \emph{Prevalence and Genetic Diversity
of the Burkholderia Bacterial Farming Symbionts in Dictyostelium Discoideum Social Amoeba Populations and their Effect on the Amoeba Microbiome}\\
\\
\textbf{ASM South Central Branch}, Poster \hfill November 01, 2019\\
\textbf{James G. DuBose}, Hunter Olsen, Tamara S. Haselkorn. \emph{Long-term Prevalence Patterns of the Burkholderia Farming Symbiont in Dictyostelium discoideum Social Amoeba Populations}
\\
\begin{flushleft}
{\Large Grants and Funding Awards} \rule{16.51cm}{0.4pt}\\
\end{flushleft}
\textbf{Computational Biology Graduate Research Assistantship} \hfill 2022\\
Proposal: \emph{A multi-omics approach for comparing the physiological differences between slow and fast-growing bacteria}\\
Award Amount: \$4,200 over one semester\\
\\
\textbf{UCA College of Natural Sciences and Mathematics Student Research Funding} \hfill 2021\\
Award amount: \$1,000 over one semester\\
Proposal: \emph{The horizontal transmission of the Paraburkholderia bacterial farming symbiont and its effects on the microbiome of the social amoeba D. discoideum}\\
\\
\textbf{Student Undergraduate Research Fellowship (SURF)} \hfill 2019 – 2020\\
Award amount: \$4,000 over two semesters\\
Proposal: \emph{The effects of the Burkholderia bacterial symbiont on its social amoeba host’s fitness and microbiome formation}\\
\\
\textbf{Advancement of Undergraduate Research in the Sciences (AURS)} \hfill 2019\\
Award amount: \$5,000 over one semester\\
Proposal: \emph{Ecological relevance of the amoeba farming symbiosis: the prevalence of the Burkholderia bacterial symbiont in natural populations, and its effect on the amoeba microbiome}
\\
\begin{flushleft}
{\Large Teaching and Mentorship} \rule{16.51cm}{0.4pt}\\
\end{flushleft}
\\
\textbf{Teaching Assistantship}\\
\textbf{Course}: Biological Principles (Georgia Tech: BIOS 1107) \hfill Fall 2022\\
\textbf{Description}: During the Fall semester of 2022, I was a teaching assistant for the Biological 
Principles course at Georgia Tech. Aside from grading, my primary responsibilities were to plan and 
hold recitations sessions to reinforce content covered in the course.\\
\\
\textbf{Research Mentorship}\\
\textbf{Thomas B. Crook}: Undergraduate Student at the University of Central Arkansas \hfill Fall 2021 – Present\\
\textbf{Description}: Thomas is an undergrad in the Haselkorn lab. Over the past two years, I have mentored Thomas
on various bioinformatic and statistical components of his honors thesis projects, as well as on designing poster 
presentations for two academic conferences.\\
\\
\textbf{Ying Li}: Undergraduate Student at Georgia Tech \hfill Summer 2022 – Present\\
\textbf{Description}: Ying is an undergrad in the Ratcliff lab. In the summer of 2022, I started 
mentoring Ying on bioinformatic software development. In the fall of 2022, I am serving as a 
mentor for Ying's senior research course, where we are developing models to evaluate how the 
algorithmic complexity of the variant annotation program we made scales with genome size and complexity.\\

\\
\begin{flushleft}
{\Large Outreach and Volunteering} \rule{16.51cm}{0.4pt}\\
\end{flushleft}
\\
\textbf{Development of Programming Education Resources for Historically Minoritized Groups in Computing}\\
\textbf{Project Advisor}: Dr. Benjamin Rydal Shapiro \\
\textbf{Synopsis}: DataWorks is a data services provider that employs people from communities that have historically 
had little access to computational resources and education. By hiring and educating people from these groups, DataWorks
hopes to broaden access to computing and foster equitable labor practices. One of the main tasks performed by DataWorks employees 
is retrieving and organizing information from online locations, a process known as web scraping. I have been working with DataWorks 
on building educational resources that will teach the employees to how to automate these tasks using Python, as opposed to manual copying and pasting.
The primary resource is a comprehensive introductory Python course that is specifically designed for people with no computational experience. 
Since web scraping is the major task performed at DataWorks, this course teaches Python through the lens of parsing webpages into tabular data.
\\
\begin{flushleft}
{\Large Employment} \rule{16.51cm}{0.4pt}\\
\end{flushleft}
\textbf{Georgia Institute of Technology, School of Biological Sciences} \hfill August 2022 – December 2022\\
Graduate Teaching Assistant\\
\\
\textbf{Georgia Institute of Technology, School of Biological Sciences} \hfill January 2022 – May 2022\\
Graduate Research Assistant\\
\\
\textbf{Arkansas Department of Health, Public Health Laboratories} \hfill March 2021 – July 2021\\
Laboratory Technician, Molecular Biology Unit, COVID-19 Unit\\
\\
\textbf{University of Central Arkansas Tutoring Center} \hfill August 2019 – May 2021\\
Biology, Chemistry, and Mathematics Tutor\\
\\
\textbf{University of Central Arkansas, Biology Department} \hfill June 2020 – August 2020\\
Research Assistant\\
\\

\begin{flushleft}
{\Large References} \rule{16.51cm}{0.4pt}\\
\end{flushleft}
Dr. William Ratcliff\\
Associate Professor, Department of Biology\\
Georgia Institute of Technology\\
Email: william.ratcliff@biology.gatech.edu\\
\\
Dr. Tammy Haselkorn\\
Associate Professor, Department of Biology\\
University of Central Arkansas\\
Email: thasekorn@uca.edu\\
\\
Dr. Benjamin Rydal Shapiro\\
Assistant Professor, Department of Learning Sciences\\
Georgia State University\\
Email: bshapiro@gsu.edu\\
\end{multicols}
\\


\end{document}