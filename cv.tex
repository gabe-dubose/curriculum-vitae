\documentclass{article}

\usepackage[letterpaper, portrait, margin=1in]{geometry}
\usepackage{parskip}

\newcommand{\qed}{\hfill$\Box$}
\newcommand{\closeproof}{\mbox{\hspace{1em}\rule{.45em}{.45em}}}
\begin{document}

\begin{flushleft}
{\Huge James G. DuBose} \rule{16.51cm}{0.4pt} \\
\end{flushleft}
\noindent
Ph.D. Student \hfill jdubos2@emory.edu\\
Population Biology, Ecology, and Evolution \hfill gabe.dubose.sci@gmail.com\\
Emory University \hfill 903-946-6255 \\
\end{flushleft}

\begin{flushleft}
{\Large Education} \rule{16.51cm}{0.4pt}\\
\end{flushleft}

\noindent Georgia Institute of Technology \hfill December 2022 \\
\noindent M.S. in Bioinformatics\\
\\
\noindent
\noindent University of Central Arkansas \hfill May 2021 \\
\noindent B.S. in Biology \\
\noindent Minors: Chemistry and Anthropology\\

\begin{flushleft}
{\Large Appointments} \rule{16.51cm}{0.4pt}\\
\end{flushleft}

\textbf{NSF Graduate Research Fellow} \hfill 2023 – Present \\
National Science Foundation/Emory University
\\
\\
\textbf{Graduate Teaching Assistant} \hfill 2023 – Present \\
Emory University
\\
\\
\textbf{Graduate Research Assistant} \hfill 2021 - 2022 \\
Georgia Institute of Technology
\\
\\
\textbf{Graduate Teaching Assistant} \hfill 2022 \\
Georgia Institute of Technology
\\
\\
\textbf{ADS Student Undergraduate Research Fellow} \hfill 2019 – 2020 \\
Arkansas Department of Higher Education
\\
\begin{flushleft}
{\Large Research Synopsis} \rule{16.51cm}{0.4pt}\\
\end{flushleft}
My primary research interest is in understanding the generation of biodiversity and biological complexity,
and I approach this through studying what facilitates and constrains evolutionary change. I have explored
this interest in several topics, but I mostly study life cycle evolution and the evolution of endosymbiotic
interactions. I like to approach my work from both genetic and ecological perspectives. While my primary
focus is on understanding the generation of biodiversity, I am also interested in the conservation of said 
biodiversity. Here, I combine my interests in life cycle evolution and conservation to study the consequences
of anthropogenic environmental and ecological change on (seasonal) phenological dynamics. 
\\

\begin{flushleft}
{\Large Publications} \rule{16.51cm}{0.4pt}\\
\end{flushleft}

\hangindent=0.7cm \textbf{DuBose, J.G.}, de Roode, J.C., The link between gene duplication and divergent patterns of gene expression across a complex life cycle, \emph{Evolution Letters}, 2024. https://doi.org/10.1093/evlett/qrae028 \\

\hangindent=0.7cm \textbf{DuBose, J.G.}, de Roode, J.C., Extensive transcriptional differentiation and specialization of a parasite across its host's metamorphosis, \emph{bioRxiv}, 2024. https://doi.org/10.1101/2024.07.16.603694 - In review at the \emph{International Journal for Parasitology} \\

\hangindent=0.7cm \textbf{DuBose, J.G.}, Crook, T.B., Matzkin, L.M., Haselkorn, T.S., The relative importance of host phylogeny and dietary convergence in shaping the bacterial communities hosted by several Sonoran Desert Drosophila species, \emph{bioRxiv}, 2024. https://doi.org/10.1101/2024.05.31.596909 - In review at the \emph{Journal of Evolutionary Biology} \\

\hangindent=0.7cm \textbf{DuBose, J.G.}, Hoogshagen, M., de Roode, J.C., The role of a non-native host plant in altering the seasonal dynamics of monarch development, \emph{bioRxiv}, 2024. https://doi.org/10.1101/2024.08.23.609406 - Submitted to \emph{Ecological Entomology} \\

\hangindent=0.7cm Pentz, J.T., MacGillivray, K., \textbf{DuBose, J.G.}, Conlin, P.L., Reinhardt, E., Libby, E., Ratcliff, W.C., Evolutionary consequences of nascent multicellular life cycles, \emph{eLife}, 2023. https://doi.org/10.7554/eLife.84336 \\

\hangindent=0.7cm \textbf{DuBose, J.G.}, Robeson, M.S., Hoogshagen, M., Olsen, H., Haselkorn, T.S., Complexities of Inferring Symbiont Function: Paraburkholderia Symbiont Dynamics in Social Amoeba Populations and Their Impacts on the Amoeba Microbiota, \emph{Applied and Environmental Microbiology}, 2022. https://doi.org/10.1128/aem.01285-22 \\

\begin{flushleft}
{\Large Teaching} \rule{16.51cm}{0.4pt}\\
\end{flushleft}
\\
\textbf{Graduate Teaching Assistant, Foundations of Modern Biology} \hfill Fall 2024 \\
Emory University: BIOL 141\\
Responsibilities: Lecturing, office hours, grading 
\\
\\
\textbf{Instructor, Microbial Ecology} \hfill Spring 2024 \\
Emory University: BIOL 470W/IBS 539\\
Responsibilities: Course design, primary instruction, lecturing, discussion leading
\\
\\
\textbf{Graduate Teaching Assistant, Foundations of Modern Biology} \hfill Fall 2023 \\
Emory University: BIOL 141\\
Responsibilities: Lecturing, office hours, grading 
\\
\\
\textbf{Graduate Teaching Assistant, Biological Principles} \hfill Fall 2022\\
Georgia Institute of Technology: BIOS 1107 \\
Responsibilities: Office hours, supplemental instruction, grading
\\
\begin{flushleft}
{\Large Talks and Presentations} \rule{16.51cm}{0.4pt}\\
\end{flushleft}
\textbf{The 3rd Joint Congress on Evolutionary Biology}, Talk \hfill July 29, 2024\\
\textbf{James G. DuBose}. \emph{The role of gene duplication in facilitating divergent patterns of gene expression across the monarch butterfly metamorphosis}\\
\\
\textbf{Front Range Microbiome Symposium 2023}, Poster \hfill April 28, 2023\\
\textbf{James G. DuBose}, Thomas B. Crook, Luciano Matzkin, Tamara S. Haselkorn. \emph{Exploring the contributions of host evolutionary history and diet in shaping the gut microbiota of cactophilic flies}\\
\\
\textbf{ASM South Central Branch 2022}, Poster \hfill October 27, 2022\\
Thomas B. Crook, \textbf{James G. DuBose}, Luciano Matzkin, Tamara S. Haselkorn. \emph{Comparative Microbiome Analysis of Cactophilic Drosophila Species}\\
\\
\textbf{Arkansas INBRE 2022}, Poster \hfill October 21, 2022\\
Thomas B. Crook, \textbf{James G. DuBose}, Luciano Matzkin, Tamara S. Haselkorn. \emph{The Microbiota of Naturally Acquired Cactophilic Drosophila Species}\\
\\
\textbf{Evolution 2021}, Talk \hfill June 23, 2021\\
\textbf{James G. DuBose}, Tamara S. Haselkorn. \emph{The transmission and diversity of Paraburkholderia in natural D. discoideum populations and its impact on the D. discoideum microbiome}\\
\\
\textbf{Asilomar 2021}, Talk \hfill January 08, 2021\\
\textbf{James G. DuBose}, Tamara S. Haselkorn. \emph{The Domination of Paraburkholderia in the Social Amoeba D. discoideum microbiome and its Impact on the Ecological Relevance of the Farming Symbiosis}\\
\\
\textbf{Arkansas INBRE 2020}, Talk \hfill November 06, 2020\\
\textbf{James G. DuBose}, Tamara S. Haselkorn. \emph{The Genetic Diversity of Bacterial Symbionts in Dictyostelium discoideum Social Amoeba and Their Effect on the Amoeba Microbiome}\\
\\
\textbf{ASM Microbe}, Poster \hfill July 2020\\
\textbf{James G. DuBose}, Hunter Olsen, Tamara S. Haselkorn. \emph{Prevalence and Genetic Diversity
of the Burkholderia Bacterial Farming Symbionts in Dictyostelium Discoideum Social Amoeba Populations and their Effect on the Amoeba Microbiome}\\
\\
\textbf{ASM South Central Branch}, Poster \hfill November 01, 2019\\
\textbf{James G. DuBose}, Hunter Olsen, Tamara S. Haselkorn. \emph{Long-term Prevalence Patterns of the Burkholderia Farming Symbiont in Dictyostelium discoideum Social Amoeba Populations}
\\
\begin{flushleft}
{\Large Grants and Funding Awards} \rule{16.51cm}{0.4pt}\\
\end{flushleft}
\textbf{NSF Graduate Research Fellowship} \hfill 2023-2028\\
Award: \$159,000\\
Proposal: \emph{Investigating heritable symbiont-mediated adaptation to climate change}\\
\\
\textbf{Computational Biology Graduate Research Assistantship} \hfill 2022\\
Award: \$4,200\\
Proposal: \emph{A multi-omics approach for comparing the physiological differences between slow and fast-growing bacteria}\\
\\
\textbf{UCA College of Natural Sciences and Mathematics Student Research Funding} \hfill 2021\\
Award: \$1,000\\
Proposal: \emph{The horizontal transmission of the Paraburkholderia bacterial farming symbiont and its effects on the microbiome of the social amoeba D. discoideum}\\
\\
\textbf{Advancement of Undergraduate Research in the Sciences (AURS)} \hfill 2019\\
Award: \$5,000\\
Proposal: \emph{Ecological relevance of the amoeba farming symbiosis: the prevalence of the Burkholderia bacterial symbiont in natural populations, and its effect on the amoeba microbiome}
\\

\begin{flushleft}
{\Large Outreach and Volunteering} \rule{16.51cm}{0.4pt}\\
\end{flushleft}
\\
\textbf{US Fish and Wildlife Service Monarch Butterfly Festival}\\
Each year, the US Fish and Wildlife Service hosts an education-oriented festival in St. Marks, Florida, where
monarchs are captured and tagged for research purposes. Each year, the de Roode lab participates with our own 
educational booth where we discuss and screen for monarch parasites with the general public. \\
\\
\textbf{Rosalynn Carter Butterfly Trail}\\
The Rosalynn Cater Butterfly Trail is a program that aims to increase habitat for native pollinators. I am frequently
invovled in various programs and events organized by the Rosalynn Cater Butterfly Trail, including their annual Spring
symposium that is focused on communicating best practices in pollinator habitat construction, as well as various projects
that involve planting said habitats. \\
\\
\textbf{Programming Education Resources for Historically Minoritized Groups in Computing}\\
In collaboration with DataWorks, a data service provider that employs people from communities that have historically 
had less access to computational resources and education, I developed and taught an introductory Python course that was specifically
designed for people with no prior computational experience.
\\
\\
\begin{flushleft}
{\Large Employment} \rule{16.51cm}{0.4pt}\\
\end{flushleft}
\textbf{Emory University} \hfill January 2023 – Present\\
Department of Biological Sciences \\
\\
\textbf{Georgia Institute of Technology} \hfill January 2022 – December 2022\\
School of Biological Sciences \\
\\
\textbf{Arkansas Department of Health} \hfill March 2021 – July 2021\\
Public Health Laboratories: Molecular Biology Unit, COVID-19 Unit\\
\\
\textbf{University of Central Arkansas} \hfill August 2019 – May 2021\\
Tutoring Center\\
\\
\textbf{University of Central Arkansas} \hfill June 2020 – August 2020\\
Biology Department\\

\begin{flushleft}
{\Large References} \rule{16.51cm}{0.4pt}\\
\end{flushleft}
Dr. Jaap de Roode\\
Professor, Department of Biology\\
Emory University\\
Email: jderood@emory.edu\\
\\
Dr. Tammy Haselkorn\\
Associate Professor, Department of Biology\\
University of Central Arkansas\\
Email: thasekorn@uca.edu\\
\\
Dr. William Ratcliff\\
Associate Professor, Department of Biology\\
Georgia Institute of Technology\\
Email: william.ratcliff@biology.gatech.edu\\
\end{multicols}
\\


\end{document}